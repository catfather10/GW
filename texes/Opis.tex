\documentclass{minimal}
\usepackage[T1]{fontenc}
\usepackage[polish]{babel}
\usepackage[utf8]{inputenc}
\usepackage{amsmath}
\begin{document}
Korzystałem głównie z pracy \textbf{Hubble without the Hubble:
Cosmology using advanced gravitational-wave detectors alone} i do niej odnoszą się numery wzorów.
Przyjąłem wartości stałych: $D_{horizon} = 500 Mpc\ \Omega_m = 0.27$, $\Omega_\Lambda = 1-\Omega_m =0.73$ , $H_0=70.4 \frac{km}{s\cdot Mpc}$.
\newline
Punktem wyjścia jest wzór (24)
\begin{equation*}
\frac{d^4N}{dt d\Theta dz d \mathcal{M}}=\frac{dV_c}{dz}\frac{\dot{n}(z)}{1+z} \mathcal{P}(\mathcal{M})\mathcal{P}(\Theta)
\end{equation*}
gdzie $\dot{n}(z)$ to \textit{merger rate density}. Całkując po masach i parametrze kątowym $\Theta$, pomijając zależność od czasu i korzystając z tego, że $:\frac{dV_c}{dz}= \frac{4\pi D_c(z)^2 D_H}{E(z)}$ (21) dostaję rozkład w zależności od $z$
\begin{equation*}
\frac{dN}{dz}=\frac{4\pi D_c(z)^2 D_H}{E(z)}\frac{\dot{n}(z)}{1+z} 
\end{equation*}
Aby z powyższego otrzymać prawdopodobieństwo $\frac{dP}{dz}$, dla danej masy $M_z$ wyznaczam $z_{max}$ rozwiązując równanie:
\begin{equation*}
\frac{A\cdot m(1+z)^{5/6}}{D_L(z)}-2=0
\end{equation*}
Po normalizacji przez całkę od $0$ do $z_{max}$, dostaję gęstość prawdopodobieństwa $\frac{dP}{dz}$, zgodnie z którą później będą losowane przesunięcia ku czerwieni. Dalej wylosowane przesunięcia ku czerwieni zamieniam w odległości jasnościowe $D_L$ korzystając z (19) i (20):
\begin{equation*}
D_L(z)=(1+z) D_c(z)=(1+z)D_H \int_{0}^{z}\dfrac{dz^\prime}{E(z^\prime)}
\end{equation*}
\begin{equation*}
E(z)=\sqrt{\Omega_m(1+z)^3+\Omega_\Lambda},\ D_H=\frac{c}{H_0}
\end{equation*}



\end{document}