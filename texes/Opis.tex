\documentclass{article}
\usepackage[T1]{fontenc}
\usepackage[polish]{babel}
\usepackage[utf8]{inputenc}
\usepackage{amsmath}
\usepackage{hyperref}
\begin{document}
\title{Rozkład jasności źródeł fal grawitacyjnych - opis}
\maketitle
\tableofcontents
Przyjęte wartości stałych:
\begin{itemize}
\item $\Omega_m = 0.27$
\item $\Omega_\Lambda = 1-\Omega_m =0.73$
\item $H_0=70.4 \frac{km}{s\cdot Mpc}$
\end{itemize}
\section{Losowanie zdarzeń}
\subsection{Losowanie położenia na niebie}
Niech $U\in[0;1]$ będzie zmienna losową z jednostajnego ciągłego rozkładu. Losowane są 4 kąty:
\begin{itemize}
\item$cos\theta=2U-1$
\item$\phi=2\pi U$
\item$cos\iota=2U-1$
\item$\psi=2\pi U$
\end{itemize}
\subsection{Losowanie masy}
Źródłem mas jest plik z \textit{Synthetic Universe}: Double Compact Objects $\triangleright$ Local $\triangleright$ Standard $\triangleright$ ABHBH02.dat .
\subsection{Losowanie przesunięcia ku czerwieni}
Punktem wyjścia jest wzór \cite[,Wzór 24]{hubble}

\begin{equation}
\frac{d^4N}{dt d\Theta dz d \mathcal{M}}=\frac{dV_c}{dz}\frac{\dot{n}(z)}{1+z} \mathcal{P}(\mathcal{M})\mathcal{P}(\Theta)  
\end{equation}
gdzie $\dot{n}(z)$ to \textit{merger rate density}. Całkując po masach i parametrze kątowym $\Theta$, pomijając zależność od czasu i korzystając z tego, że $:\frac{dV_c}{dz}= \frac{4\pi D_c(z)^2 D_H}{E(z)}$ \cite[,Wzór 21]{hubble} dostaję rozkład w zależności od $z$
\begin{equation}
\frac{dN}{dz}=\begin{cases}\frac{4\pi D_c(z)^2 D_H}{E(z)}\frac{\dot{n}(z)}{1+z}  & z\in[0;z_{max}]\\
0 & z\notin[0;z_{max}]
\end{cases}.
\end{equation}
Aby z powyższego otrzymać prawdopodobieństwo $\frac{dP}{dz}$, dla największej masy $M$ wyznaczane jest $z_{max}$ rozwiązujące równanie:
\begin{equation}
\frac{A (M(1+z))^{5/6}}{D_L(z)}-\frac{\mbox{ SNR}_{min}}{4}=0
\end{equation}
dla $\mbox{ SNR}_{min} = 8$. Dzielenie $\mbox{ SNR}_{min}$ przez 4 w powyższym wynika z czynnika kątowego $\Theta \in[0;4] $. Po normalizacji przez całkę od $0$ do $z_{max}$, otrzymywana jest gęstość prawdopodobieństwa $\frac{dP}{dz}$. Wyznaczane jest maksymalne prawdopodobieństwo $ \frac{dP_{max}}{dz}$ używane do ograniczenia obszaru losowania $z$ metodą Monte Carlo.
\subsection{Generowanie próbki}
Z wylosowanych przesunięć ku czerwieni obliczane są odległości jasnościowe $D_L$ korzystając z \cite[,Wzór 19]{hubble} i \cite[,Wzór 20]{hubble}:
\begin{equation}
D_L(z)=(1+z) D_c(z)=(1+z)D_H \int_{0}^{z}\dfrac{dz^\prime}{E(z^\prime)}
\end{equation}
\begin{equation}
E(z)=\sqrt{\Omega_m(1+z)^3+\Omega_\Lambda},\ D_H=\frac{c}{H_0}
\end{equation}

Wylosowane zdarzenie jest zapisywane jeśli jego $\mbox{SNR} \geq \mbox{SNR}_{min}=8$
\begin{equation}
\mbox{SNR}=\frac{A\Theta(\theta, \psi, \iota, \psi) M_z^{5/6}}{D_L}
\end{equation}
\section{Metoda największej wiarygodności}
Porównywane są: model $\alpha=0$ : $\dot n (z)=const $ i model $\alpha=1$ : $\dot n (z)=1+z $. Z wcześniej wygenerowanej próbki 100 000 zdarzeń otrzymywana jest analityczna postać $\left.\frac{dP}{d\mbox{SNR}} \right |_{\alpha=1}$ i $\left.\frac{dP}{d\mbox{SNR}} \right |_{\alpha=0}$ za pomocą przybliżenia $\frac{dP}{d\mbox{SNR}}\approx\frac{1}{N}\frac{N_i}{\Delta H}$, gdzie $N$ to całkowita wielkość próbki,a $N_i$ to wysokość binu o szerokości $\Delta H$. Wiarygodność (\textit{likelihood}) to $\mathcal{L}=\prod_{i=0}^{n}\frac{dP}{d\mbox{SNR}}(x_i)$ dla $n\in\{3,6,10,30,60,100\}$. Następnie obliczany jest \textit{Bayes factor} $ \mathcal{O}=\frac{\mathcal{L}(\alpha =1)}{\mathcal{L}(\alpha =0)}$ dla 10 000 próbek o wielkości $n$.


\begin{thebibliography}{9}
\bibitem{hubble}
Hubble without the Hubble: cosmology using advanced gravitational-wave detectors alone
\url{https://arxiv.org/abs/1108.5161v2}
\end{thebibliography}

\end{document}
